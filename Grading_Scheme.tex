\documentclass{article}
\usepackage[utf8]{inputenc}
\usepackage{amsmath}
\usepackage{amssymb}
\usepackage{graphics}
\usepackage{graphicx}
\title{Grading scheme}
\author{Dead TA's Society}
\date{Spring 2023}


\begin{document}

\maketitle

$$\text{Grade } = \frac{5}{n}\sum_{i=1}^{n}s_i(p_i+a_i)$$
Where,
\begin{itemize}
    \item $n$ is the total number of recitations.
    \item $s_i$ is submision score for the $i$th recitation, they have to show me that they have attempted problems either in recitation or show me later, either in person or through teams. \\
    $ 0\leq s_i \leq 1 $, $ s_i = 1 $ if they have attempted as many problems as we covered in recitation (they can gain this by just sitting in recitation and doing problems that I give them there). \\
    $ s_i=0 $ is no problems are attempted and $ 0 < s_i < 1 $ if some problems are attempted but not as many as we covered in recitation. $ s_i = 1 $ is that all problems are attempted that were covered in the recitation.
    \item $ a_i $ is attendance score for the $ i $th recitation this can be gained by just showing up $ 0\leq a_i \leq 0.25 $. 
    \item $ p_i $ is the participation score for the $ i $th recitation where $ 0\leq p_i+a_i \leq 1 $. This can be gained in 2 ways either paying attention and engaing in recitation or you can do few more problems than the problems covered in recitations and show me in person or submit online (through teams). \\
    This is so attendance is not mandatory but doing the work is, so if you do a lot of problems on your own and show me but don't attend the recitation you can still get the score. Also just attending doesn't mean you get points if you are attending and not doing any work.
\end{itemize}




\end{document}