\documentclass{exam}
\usepackage[utf8]{inputenc}
\usepackage{amsmath}
\usepackage{amssymb}
\usepackage{graphics}
\usepackage{graphicx}
\title{Number theory Problems}
\author{CS/MATH 113 team}

\printanswers 
\begin{document}

\maketitle
The questions with (*) are hard, the ones with (+) are medium level difficulty, the ones with (-) are easy level the ones (**) are very hard and not doable by students
\begin{questions}
    \question (*) Prove that for all natural numbers $n>1$, $ \sqrt[n]{n}$ is irrational
    \begin{solution}
        Suppose $\sqrt[n]{n}$ is rational for some $n \in \mathbb{N}$
        \\Then there exists integers $a$ and $b$, such that $\sqrt[n]{n} = \frac{a}{b}$, where $b \neq 0$ and $gcd(a,b) =1$
        $$\sqrt[n]{n} = \frac{a}{b} \Rightarrow n = \frac{a^n}{b^n}$$
        $$gcd(a,b) =1 \Rightarrow gcd(a^n,b^n) =1$$
        As $n \in \mathbb{N}$, then $b^n = 1$, which means $n = a^n$
        \\As $n > 0$ and $b^n = 1$, then $a^n > 0$, which means that $a > 0$
        \\$a \neq 1$, as if $a = 1$ then $n = \frac{a^n}{b^n} = \frac{1}{1} = 1$, but $n > 1$, so $a \geq 2$
        \\We know for all natural numbers $n$ $2^n > n$ (this result is trivial and can be easily proved by mathematical induction.)
        \\So $a^n \geq 2^n > n$, which means $n \neq a^n$, there we have a contradiction with out orignal claim that $n = a^n$
        \\Therefore for all natural numbers $n>1$, $ \sqrt[n]{n}$ is irrational
        \begin{flushright}
            $\square$
        \end{flushright}
    \end{solution}

    \question (*) Given that $p$ is a prime and $p|a^n$, prove that $p^n|a^n$.
    \begin{solution}
        As $p|a^n$ then $a^n = kp$ for some integer $k$.
        \\\textbf{Case 1:} $p \neq a$
        \\Then $a$ is not a prime, then $a = p_1\times p_2 \times ... p_m$
        \\$a^n = p_1^n\times p_2^n \times ... p_m^n = kp$
        \\As $p|a^n$ and $a^n = p_1^n\times p_2^n \times ... p_m^n$ then there must be some $p_i$ from $1\leq i \leq m$ such that $p|p_i$
        \\As $p_i$ is prime for all $i \leq i \leq m$, then if $p|p_i$ then $p_i = p$ which means $p|a$
        \\Then $a = pq$ so $a^n = p^n q^n$ therefore $p^n|a^n$.
        \\\textbf{Case 2:} $p = a$ 
        \\If $p = a$  and $p|a^n$ then as $a^n|a^n$ and $a^n=p^n$ then $p^n|a^n$.
        \begin{flushright}
            $\square$
        \end{flushright}
    \end{solution}
    \pagebreak

    \question (+) Show that any composite three-digit number must have a prime factor less than or equal to 31.
    \begin{solution}
        The next prime after 31 is 37, then the smallest composite number not containing a prime factor less than or equal to 31 would be $37^2 = 1369$ which is 4 digits.
        \begin{flushright}
            $\square$
        \end{flushright}
    \end{solution}

    \question (*) Show that $\sqrt{p}$ is irrational for any prime number $p$.
    \begin{solution}
        Suppose $\sqrt{p}$ is rational then $\sqrt{p} = \frac{r}{q}$ where $q \neq 0$ and $gcd(q,r) = 1$
        \\Then $p = \frac{r^2}{q^2}$, so $pq^2 = r^2$
        \\Now as $r^2 = r \times r$ then any number in prime factorization of $r^2$ would appear an even number of times. 
        \\Similiary any number in prime factorization on $q^2$ appear and even number of times.
        \\So take $q^2 = p_1 \times p_2 \times... p_n \times p_1 \times p_2 \times... p_n$
        \\As $p|r^2$ and $q^2|r^2$ then $r^2 = p \times p_1 \times p_2 \times... p_n \times p_1 \times p_2 \times... p_n$
        \\Now $p$ is a number that appears in prime factorization of $r^2$ an odd number of times.
        \\Here we have a contradiction, therefore $\sqrt{p}$ is irrational.
        \begin{flushright}
            $\square$
        \end{flushright}
    \end{solution}
    
    \question (+) Show that if $a$ is a positive integer and $\sqrt[n]{a}$ is rational, then $\sqrt[n]{a}$ must be an integer.
    \begin{solution}
        Let $a \in \mathbb{Z}^+$, suppose $\sqrt[n]{a}$ is rational, we show that then $\sqrt[n]{a}$ must be an interger.
        \\Let $\sqrt[n]{a} = \frac{p}{q}$, where $p,q \in \mathbb{Z}$ where $q \neq 0$ and $\text{gcd}(p,q) = 1$.
        $$\sqrt[n]{a} = \frac{p}{q} \Leftrightarrow a = \frac{p^n}{q^n} \Leftrightarrow a q^n = p^n$$
        Now we have that $q^n | p^n$, but as $\text{gcd}(p,q) = 1$ then $\text{gcd}(p^n,q^n) = 1$.
        \\So as only common divider of $p^n$ and $q^n$ is 1 and $q^n | p^n$ then $q^n = 1$
        \\Therefore $ a = \frac{p^n}{q^n} = p^n$, so $\sqrt[n]{a} = p$.
        \\Which means $sqrt[n]{a}$ is an integer.
        \begin{flushright}
            $\square$
        \end{flushright}
    \end{solution}
    
    \question (**) In this question we will prove Euclid's Lemma that if $p$ is a prime number that divides $ab$ then $p$ divides $a$ or $p$ divides $b$.
    \\We shall prove this by proving a lemma and using a corollary from that lemma.
    \\\textbf{Well ordering principle:} Every non empty set of positive integers have a smallest element.
    \\\textbf{Division algorithm:} if $a,b \in \mathbb{Z}$, where $b>0$, then there exists unique $q,r \in \mathbb{Z}$, $a=bq+r$ where, $0 \leq r <b$ 
    \begin{parts}
    \part \textbf{Bezout's lemma:} for all integers $a$ and $b$ there exist integers $s$ and $t$ such that $gcd(a,b) = as + bt$
    \begin{solution}

        Let $S =\{am + bn \mid m,n\in \mathbb{Z} \mbox{ and } am+bn>0\}$
        \\Due to well ordering principle $S$ has a smallest element $d$
        $$d = as + bt$$
        We claim that $d=gcd(a,b)$
        \\Using the division algorithm $a = dq+r$, where $0\leq r<d$
        \\We assume $r>0$, and reach a contradiction, from which we can conclude that $r=0$ thus $d$ would divide $a$
        \\If $r>0$
        $$r = a - dq = a - (as + bt)q = a - asq - btq = a(1-sq)+b(-tq) \in S$$
        $r$ is in the form that it belongs to our set $S$, but as said above $r<d$ thus it contradicts the fact that $d$ is the smallest element in $S$
        \\Thus $r=0$, which means $d$ divides $a$
        \\Same argument can be constructed for $b$ and used to show that $d$ divides $b$ as well.
        \\Now assume there exist $d'$ that is also a divisor of $a$ and $b$.
        \\Let $a=d'h$ and $b = d'k$
        \\Then $d = as+bt= (d'h)s+(d'k)t=d'(sh +kt)$, then $d'$ is also a divisor of $d$
        \\Thus $d>d'$, so by universal generalization we can conclude that $d$ is the greatest of all divisors of $a$ and $b$.
        Thus contradiction with the fact that $d$ is the smallest element.
        \begin{flushright}
            $\square$
        \end{flushright}
    \end{solution}

    \part \textbf{Corollary of bezout's lemma:} If $a$ and $b$ are relatively prime then $as+bt = 1$

    \part Using the above corollary prove Euclid's lemma.
        \begin{solution}
            Let $p$ be a prime that divides $ab$ but does not divide $a$
            \\We need to show that $p$ must divide $b$
            \\As $p \nmid a$ and $p$ is a prime then $\text{gcd}(a,p) = 1$
            \\Then there exist $s,t\in \mathbb{Z}$ such that $1=as +pt$
            $$b=abs +pbt$$
            as $p$ divides right hand side then $p$ would divide $b$ as well.
            \begin{flushright}
                $\square$
            \end{flushright}
        \end{solution}
    \end{parts}

    \pagebreak
    \question (*) For all positive integers $a$ and $b$ show that $\text{gcd}(a,b) \text{lcm}(a,b)=ab$.
    \begin{solution}
        Let $d = \text{gcd}$ for $a,b \in \mathbb{Z}$. Then $\exists p,q \in \mathbb{Z}$ s.t. $a = pd$ and $b = qd$.
        \\Let $m = \frac{ab}{d}$ then $m = aq = pb$. Which means $a|m$ and $b|m$ which mean $m$ is a common multiple of $a$ and $b$.
        \\Now we need to show that $m$ is indeed the least common multiple of $a$ and $b$.
        \\Let $c$ be a common multiple of $a$ and $b$, then $c = at = sb$.
        \\From bezout's lemma we know that $\exists x,y \in \mathbb{Z}$ s.t. $d = ax + by$.
        \\We show that $m|c$ which would imply that $m \leq c$.
        $$\frac{c}{m} =  \frac{cd}{ab} = \frac{c(ax + by)}{ab} = \frac{cax}{ab} + \frac{cby}{ab}$$
        $$\frac{cax}{ab} + \frac{cby}{ab} = \frac{cx}{b} + \frac{cy}{a} = \frac{c}{b}x + \frac{c}{a}y$$
        $$\frac{c}{m} = \frac{c}{b}x + \frac{c}{a}y = sx + ty$$
        As $s,x,t,y \in \mathbb{Z}$ then $sx + ty \in \mathbb{Z}$, which means $m|c$ theefore $m \leq c$.
        \\Which means $m$ is the least common multiple of $a$ and $b$.
        \\So we have that $dm = \text{gcd}(a,b) \text{lcm}(a,b) = ab$.
        \begin{flushright}
            $\square$
        \end{flushright}
    \end{solution}

    \question (*) Show that there are infinitely many primes, in other words the set containing all prime numbers is infinite.
    \\\textbf{Definition:} A prime number is a Natural number that is only divisible by 1 and itself, and has to be divisible by 2 different numbers.
    \\\textbf{Fundamental Theorem of Arithmetic:} Every integer $N > 1$ has a prime factorization, meaning either $N$ is itself prime or can be written as a product of prime numbers.
    \begin{solution}
        Let $s=\{p_0,p_1,p_2,...,p_n\}$ be set of all primes. 
        \\Let $P = p_0 \times p_1 \times p_2 \times ... \times p_n$
        \\Let $q = P+1$
        \\\textbf{Case 1:}
        \\$q$ is prime, which is not in our set $s$
        \\\textbf{Case 2:} 
        \\if $q$ is not prime, then there exits a prime factor decomposition of $q$.
        \\Let $f$ be a prime that divides $q$, then $f$ would be in our set $s$ thus $f$ would divide $P$ too. 
        \\As $f$ divides $q$ and $P$ then $f$ divides $q-P$, which is $1$
        \\Then $f$ divides 1.
        \\As $f\geq2$ $f$ cannot divide 1, thus we have a contradiction.
        \begin{flushright}
            $\square$
        \end{flushright}
    \end{solution}

    \question (+) Prove the following claim: There exists irrational numbers $a$ and $b$ such that $a^b$ is rational.
    \begin{solution}
        Take $a = \sqrt{2}$ and $b = \sqrt{2}$
        $$c = a^b$$
        \textbf{Case 1:}
        \\If $\sqrt{2} ^{\sqrt{2}}$ is rational then we already have our irrational numbers $a$ and $b$ such that $a^b$ is rational
        \\\textbf{Case 2:}
        \\If $\sqrt{2} ^{\sqrt{2}}$ is irrational then, let $a = \sqrt{2} ^{\sqrt{2}}$ and $b = \sqrt{2}$
        $$c = \left(\sqrt{2} ^{\sqrt{2}}\right)^{\sqrt{2}} = 2$$
        and 2 is rational
        \begin{flushright}
            $\square$
        \end{flushright}
    \end{solution}

    \question (+) Show that $\sqrt{2}$ is irrational. In other words, $\sqrt{2}$ cannot be written in the form $\frac{p}{q}$ where $p,q \in \mathbb{Z}$ and $q \neq 0$
    \begin{solution}
        Assume $\sqrt{2}$ is rational, then $\sqrt{2} = \frac{p}{q}$, where $p,q \in \mathbb{Z}$ and $q \neq 0$.
        \\And $\frac{p}{q}$ is the lowest form it can be. 
        $$\left(\frac{p}{q}\right)^2 = 2$$
        $$p^2 = 2 q^2$$
        This implies $p$ is even which means $p = 2k$, for some $k \in \mathbb{Z}$
        $$4k^2 = 2 q^2$$
        $$2k^2 = q^2$$
        This implies $q$ is even.
        \\But $p$ and $q$ can't both be even as they are in the lowest form possible thus the 2 would be canceled. 
        \\Here we have a contradiction.
        \\Thus $\sqrt{2}$ cannot be written in form $\frac{p}{q}$ where $p,q \in \mathbb{Z}$
        \\Thus $\sqrt{2}$ is irrational.
        \begin{flushright}
            $\square$
        \end{flushright}
    \end{solution}

    


\question (-) Explain what you must do to disprove the statement:
  $x^3+5x + 3$ has a root between $x = 0$ and $x=1$
  \begin{solution}
  The statement in logical notation is 
  $$\exists x \text{ such that } (0<x<1 \land x^3+5x+3 = 0)$$
  Giving a counterexample is not enough. Saying that when $x=0.5$ then $x^3 +5x+3 \neq 0$ is not sufficient. \\
  To disprove this statement, we need to prove that the \textbf{negation is true} which is
  $$\neg \exists x \text{ such that } 0<x<1 \land x^3+5x+3 = 0 \equiv \forall x \text{ such that } \neg(0<x<1 \land x^3+5x+3 = 0)$$
  Or in English
  \begin{center}
      For all $x$, it is not the case that both $x$ is between 0 and 1 and $x^3+5x+3=0$
  \end{center}
      \end{solution}

\question (-) Prove that for any integer $n$ the number $n^2+5n+13$ is odd
\begin{solution}

If $n$ is an integer, it can either be even or odd.\\
\textbf{Case 1:} $n$ is even. Therefore $n=2a, a\in \mathbb{Z}$
\begin{align*}
    &(2a)^2 + 5(2a) + 13 \\
    =&4a^2 + 10a + 13 \\
    =&4a^2+10a + 12 + 1 \\
    =&2(2a^2 + 5a+6) + 1
\end{align*}
Therefore $n^2+5n+13$ is odd in this case. \\
\textbf{Case 2:} $n$ is odd. Therefore $n=2a+1, a\in \mathbb{Z}$
\begin{align*}
    &(2a+1)^2 + 5(2a+1) + 13 \\
    =&4a^2 + 4a+1 + 10a+5 + 13 \\
    =&4a^2+14a + 19 \\
    =&4a^2+14a + 18+1 \\
    =&2(2a^2 + 7a+9) + 1
\end{align*}
Therefore $n^2+5n+13$ is odd in this case. \\

\textbf{Since the statement is true in all cases, it is true in general.}

 \end{solution}
\question (-) State the statement of Contradiction and verify that it is a valid argument.\\
\textbf{Hint:} In contradiction we are saying that $A$ implies $B$ is the same as saying that $A$ and $\neg B$ happening together is false.
\begin{solution} 
\\
Statement is 
\[(A \implies B ) \equiv ((A \land \neg B) \;\; is \;\; \text{false})\]
We can show that one side is equivalent to the other
\begin{equation*}
    \neg (A \land \neg B) \equiv (\neg A \lor B) \equiv (A \implies B)
\end{equation*}
Therefore it is true
\end{solution}


\question (-) Show through contraposition the following proposition is true: $x \in \mathbb{Z}$. If $7x + 9$ is even, then $x$ is odd.


\begin{solution}
\textbf{Proof by Contrapositive}\newline Let $P$ be " $7x+9$ is even" and $Q$ be "$x$ is odd" \newline
Instead of doing a direct proof where we show $P\implies Q$, we would show that $\neg Q \implies \neg P$ since that seems easier.\\
Suppose $x$ is not odd. \newline
Thus $x$ is even, so $x = 2a$ for some integer $a$.\newline
Then
\begin{equation}
7x + 9    
\end{equation}
\begin{equation}
 = 7(2a) + 9   
\end{equation}

\begin{equation}
  = 14a + 8 + 1  
\end{equation}

\begin{equation}
     2(7a + 4) + 1
\end{equation}
Therefore $7x + 9 = 2b + 1$, where $b$ is the integer $7a + 4$.\newline
Consequently $7x + 9$ is odd. \newline
Therefore 7x + 9 is not even \newline
Therefore proving $\neg Q \implies \neg P$ thus logically equivalent to $P \implies Q$
\end{solution}


\question (-) Prove that ``$(a+b)^2 = a^2 +b^2$'' is \textbf{not} an algebraic identity where $a,b \in \mathbb{R}$
  \begin{solution}
  We can disprove this by finding \textbf{specific} real numbers $a$ and $b$ for which the equation is false. \\
  If an equation is \textbf{not} an identity, you can usually find a counterexample by trial and error. In this case, if $a=1,b=2$ then
  \begin{center}
      $(a+b)^2 = (1+2)^2 = 3^2 = 9$ while $a^2+b^2 = 1^2+2^2 = 5$ 
  \end{center}
  So if $a=1,b=2$ then $(a+b)^2 \neq a^2+b^2$ and hence the statement is not an identity. \vspace{3mm}\\
  A common mistake is to say:
  \begin{center}
``$(a+b)^{2}=a^{2}+2 a b+b^{2}$, which is not the same as $a^{2}+b^{2} .$''
  \end{center}
In the first place, how do you know $a^{2}+2 a b+b^{2}$ is not the same as $a^{2}+b^{2} ?$ It is no answer to say that they look different - after all, $(\sin \theta)^{2}+(\cos \theta)^{2}$ looks very different than 1 , but $(\sin \theta)^{2}+(\cos \theta)^{2}=1$ is an identity. \vspace{3mm}\\
In the second place, $a^{2}+2 a b+b^{2}$ is the same as $a^{2}+b^{2}$ if (for instance) $a=17$ and $b=0$ - and they're equal for many other values of $a$ and $b$.
  \end{solution}


\question (-) Prove that for $m$ and $n$ integers, if 2 divides $m$ or 10 divides $n$, then 4 divides $m^{3}n^{2}$
\begin{solution}
 $$(m \text{ mod } 2 = 0 \lor n \text{ mod } 10 = 0) \implies m^{3}n^{2} \text{ mod } 4 = 0 $$
 \\Case 1: $m$ mod $2 = 0$ is true.
 \\This is when $m=2x$ where $x \in \mathbb{Z}$
 \\Then:
 $$(2x)^{3}n^{2}$$
 $$8x^{3}n^{2} $$
 $$4(2x^{3}n^{2}) $$
The above is divisible by 4.
\\Proved for $m$ mod 2 $= 0$.
\\Case 2:
\\$n$ mod 10 $=0$ is true:
 \\This is when $n=10x$ where $x \in \mathbb{Z}$
 \\then:
 $${m^{3}(10x)^{2}} $$
 $${m^{3}100x^{2}} $$
  $${4(25m^{3}x^{2})} $$
\\The above is divisible by 4
\\Proved for $n$ mod 10 $=$ 0.
\end{solution}




\question (-) Give a counterexample to the statement
\begin{center}
    ``If $n$ is an integer and $n^2$ is divisible by 4, then $n$ is divisible by 4''
\end{center}
\begin{solution}
To give a counterexample, we need an integer $n$ such that $n^2$ is divisible by 4 but $n$ is \textbf{not} divisible by 4 - the ``if'' part must be true, but the ``then'' part must be false. For example, $n=6$. Then $n^2=36$ is divisible by 4 but $n=6$ is not divisible by 4. Thus, $n=6$ is a counterexample to the statement. \\
Note that $n=5$ is not divisible by 4, $n^2 = 25$ is also not divisible by 4. Both the ``if'' and ``then'' parts of the statement are both false. Therefore, $n=5$ is not a counterexample to the statement.
\end{solution}


\question (-) Show through contraposition the following proposition is true : If $x^{2} - 6x + 5$ is even, then x is odd.

\begin{solution}
A direct proof seems difficult. We would begin by assuming that
$x^{2} - 6x + 5$ is even, so $x^{2} - 6x + 5 = 2a$.\newline \newline Then we would need to transform this
into $x = 2b + 1$ for $b \in \mathbb{Z}$. But it is not quite clear how that could be done,
for it would involve isolating an $x$ from the quadratic expression.\newline \newline However
the proof becomes very simple if we use contrapositive proof.\newline \newline
Proposition Suppose $x \in \mathbb{Z}$. If $x^{2} - 6x + 5$ is even, then $x$ is odd. \newline\newline
Proof. (Contrapositive) Suppose $x$ is not odd.
Thus $x$ is even, so $x = 2a$ for some integer $a$.
So
\begin{equation}
     x^{2}-6x+5
\end{equation}

\begin{equation}
  = (2a)^{2}-6(2a)+5
\end{equation}
  
\begin{equation}
    = 4a^{2}-12a+5
\end{equation}

\begin{equation}
    4a^{2} -12a+4+1
\end{equation}

\begin{equation}
      = 2(2a^{2}-6a+2)+1.
\end{equation}

Therefore $x^{2} - 6x + 5 = 2b + 1$, where b is the integer $2a^{2} - 6a + 2$ \newline \newline
Consequently $x^{2} - 6x + 5$ is odd.
Therefore $x^{2} - 6x + 5$ is not even.\newline \newline In summary, since $x$ being not odd ($\neg Q$) resulted in $x^{2} - 6x +5$ being not
even ($\neg P$), then $x^{2} - 6x + 5$ being even ($P$) means that $x$ is odd ($Q$). \newline \newline Thus
we have proved  $P \implies Q$  by proving $\neg Q \implies \neg P$
\end{solution}


    
\end{questions}

\end{document}