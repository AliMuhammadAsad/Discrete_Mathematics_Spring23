\documentclass[addpoints]{exam}

\usepackage{amsmath}
\usepackage{amssymb}
\usepackage{geometry}
\usepackage{venndiagram}
\usepackage{graphicx}
\usepackage{xcolor}
% Header and footer.
\pagestyle{headandfoot}
\runningheadrule
\runningfootrule
\runningheader{Discrete Mathematics}{Problem Set 2}{CS/Math 113}
\runningfooter{}{Page \thepage\ of \numpages}{}
\firstpageheader{}{}{}

\boxedpoints
\printanswers
\qformat{} %Comment this to number questions, uncomment this to not number questions

\newcommand\union\cup
\newcommand\inter\cap
\newenvironment{definition}[2][Definition]{\begin{trivlist}
    \item[\hskip \labelsep {\bfseries #1}\hskip \labelsep {\bfseries #2.}]}{\end{trivlist}}
\newenvironment{problem}[2][Problem]{\begin{trivlist}
    \item[\hskip \labelsep {\bfseries #1}\hskip \labelsep {\bfseries #2.}]}{\end{trivlist}}

\title{CS/Math 113 - Problem Set 1}
\author{Habib University - Spring 2023}
\date{Week 01}

\begin{document}
\maketitle
\section{Problems}
\begin{sloppypar}
\begin{problem}{1}
Let $p$ and $q$ be the proposition
\begin{itemize}
    \item[$p:$] The election is decided.
    \item[$q:$] The votes have been counted.
\end{itemize}
Express each of these compound propositions as an English sentence.

\begin{itemize}
    \item[(a)] $\neg p$
    \item[(b)] $p \lor q$
    \item[(c)] $\neg p \land q $
    \item[(d)] $q \implies p $
    \item[(e)] $ \neg q \implies \neg p $
    \item[(f)] $\neg p \implies \neg q$
    \item[(g)] $p \leftrightarrow q$
    \item[(h)] $\neg q \lor (\neg p \land q) $
\end{itemize}
\end{problem}

\begin{questions}
    \question
    \begin{solution}
        
        (a) The election is \textbf{\textit{not}} decided

        (b) The election is decided \textbf{\textit{or}} the votes have been counted

        (c) The election is \textbf{\textit{not}} decided \textbf{\textit{and}} the votes have been counted

        (d) If the votes have been counted, then the election is decided

        (e) If the votes have not been counted, then the election is not decided 

        (f) If the election is not decided, then the votes have not been counted 

        (g) The election is decided if and only if the votes have been counted

        (h) The votes have not been counted \textbf{\textit{or}} the election is \textbf{\textit{not}} decided \textbf{\textit{and}} the votes have been counted
    \end{solution}
\end{questions}

\begin{problem}{2}
Let $p$ and $q$ be the proposition
\begin{itemize}
    \item[$p:$] It is below freezing.
    \item[$q:$] It is snowing.
\end{itemize}

Write these propositions using $p$ and $q$ and logical connectives (including negations)

\begin{itemize}
    \item [(a)] It is below freezing and snowing.
    \item [(b)] It is below freezing but not snowing.
    \item [(c)] It is not below freezing and it is not snowing.
    \item [(d)] It is either snowing or below freezing (or both).
    \item [(e)] If it is below freezing, it is also snowing.
    \item [(f)] Either it is below freezing or it is snowing, but it is
          not snowing if it is below freezing.
    \item [(g)] That it is below freezing is necessary and sufficient
          for it to be snowing.
\end{itemize}

\end{problem}

\begin{questions}
    \question
    \begin{solution}
        
        (a) $ p \land q $

        (b) $ p \land \neg q $

        (c) $ \neg p \land \neg q $ 

        (d) $ q \lor p $

        (e) $ p \implies q $

        (f) $ (p \lor q) \land (p \implies \neg q)$

        (g) $ p \iff q $
    \end{solution}
\end{questions}

\begin{problem}{3}
Construct a truth table for each of these compound propositions.
\begin{itemize}
    \item[(a)] $ p \land \neg p$
    \item[(b)] $ p \lor \neg p$
    \item[(c)] $ (p \lor \neg q) \implies  q$
    \item[(d)] $ (p \lor q) \implies (p \land q) $
    \item[(e)] $ (p \implies q) \leftrightarrow (\neg q \implies \neg p) $
    \item[(f)] $ (p \implies q) \implies (p \implies q) $
\end{itemize}

\end{problem}

\begin{questions}
    \question
    \begin{solution}

        (a) $ p \land \neg p $
        \[ 
        \begin{array}{c | c || c}
            p & \neg p & p \land \neg p \\ 
            \hline
            F & T & F \\ 
            T & F & F
        \end{array}
        \]

        (b) $ p \lor \neg p$
        \[ 
        \begin{array}{c | c || c}
            p & \neg p & p \lor \neg p \\ 
            \hline
            F & T & T \\
            T & F & T
        \end{array}
        \]

        (c) $ (p \lor \neg q) \implies  q$
        \[ 
        \begin{array}{c | c | c | c || c}
            p & q & \neg q & p \lor \neg q & (p \lor \neg q) \implies q \\ 
            \hline
            F & F & T & T & F  \\ 
            F & T & F & F & T  \\ 
            T & F & T & T & F  \\ 
            T & T & F & T & T
        \end{array}
        \]

        (d) $ (p \lor q) \implies (p \land q) $
        \[ 
        \begin{array}{c | c | c | c || c}
            p & q & p \lor q & p \land q & (p \lor q) \implies (p \land q) \\ 
            \hline
            F & F & F & F & T    \\ 
            F & T & T & F & F    \\
            T & F & T & F & F    \\
            T & T & T & T & T
        \end{array}
        \]

        (e) $ (p \implies q) \leftrightarrow (\neg q \implies \neg p) $
        \[ 
        \begin{array}{c | c | c | c | c | c || c}
            p & q & \neg p & \neg q & p \implies q & \neg q \implies \neg p & (p \implies q) \leftrightarrow (\neg q \implies \neg p) \\ 
            \hline
            F & F & T & T & T & T & T      \\ 
            F & T & T & F & T & T & T      \\
            T & F & F & T & F & F & T      \\ 
            T & T & F & F & T & T & T      
        \end{array}
        \]

        (f) $ (p \implies q) \implies (p \implies q) $
        \[ 
        \begin{array}{c | c | c || c}
            p & q & p \implies q & (p \implies q) \implies (p \implies q) \\ 
            \hline
            F & F & T & T     \\ 
            F & T & T & T     \\ 
            T & F & F & T     \\ 
            T & T & T & T
        \end{array}
        \]

    \end{solution}
\end{questions}

\pagebreak
\begin{problem}{4}
In an Island there are two kinds of inhabitants, knights, who always tell the truth, and knaves, who always lie. You encounter two people
A and B. Determine, if possible, what A and B are if they address you in the ways described.
\begin{itemize}
    \item[(a)] A says ``At least one of us is a knave'' and B says nothing.
    \item[(b)] A says ``The two of us are both knights'' and B says ``A is a knave''.
    \item[(c)] A says ``I am a knave or B is a knight'' and B says nothing.
    \item[(d)] Both A and B say ``I am a knight''
    \item[(e)] A says we ``We are both knaves '' and B says nothing.
\end{itemize}
\end{problem}
\begin{questions}
    \question
    \begin{solution}
        
        (a) If A is a knave, then he is telling the truth as there does indeed exist at least one knave. But knaves always lie, so A cannot be a knave. Then A is a knight. Knights always tell a truth, therefore, there must indee be a knave amongst the two. A cannot be a knave, therefore B is a knave.

        (b) If A is a knight, then A's statement must be true. However, B's statement contradicts A's statement. So B cannot be a knight, and this case is false. However, if B is a knave, then B is telling a lie which is true, but contradicts A's statement, therefore, A's statement cannot be true. So this case is false \\ 
        If A is a knave, and B is a knave, then B is telling the truth, so this can't be the case. \\ 
        Then if A is a knave, and B is a knight, then there is no contradiction here, as A is telling a lie which holds true as a knave must always tell lies, and B is telling the truth that A is a knave. \\ 
        Therefore, A is a knave and B is a knight. 

        (c) If A is a knave, then he is telling the truth about himself, that he is a knave which implies his statement is true. But knaves always lie, so his statement should be false, therefore A cannot be a knave. Then A is a knight.\\ Now we know that A is a knight, so in A's statement, ``I am a knave'' is a not true, so for the statement to be true, ``B is a knight'' must be true. \\ Then it can be concluded that A is a knight, and B is a knight.

        (d) It cannot be determined as both statements can hold true in either scenarios. If A is a knight, he is telling the truth so is true, but if A is not a knight, the he is lying which means he is a knave, which is also true. \\ The same argument holds true for B, therefore A and B cannot be determined.

        (e) If A is a knight, then this statement is false by the rules of the knight, therefore A is not a knight. So A must be a knave. A is a knave, then A's statement must be a lie and both cannot be knaves. Since A is a knave, then B cannot be a knave so B is a knight. \\ A is a knave and B is a knight. 
    \end{solution}
\end{questions}

\pagebreak
\begin{problem}{5}
Three friends are hanging out at a Cafe. The server approaches them and ask ``Who wants a slice of cake ?'' The first friend says ``I don't know''.
The second friend says ``I don't know''. Finally, the third friend says ``No, not everyone wants cake''. The serves comes back and gives slices of cake to the
friends who want . How did the server figure out who wanted the cake ? 
\end{problem}
\begin{questions}
    \question
    \begin{solution}

        (If the server was this smart then that server would never be serving in the first place. \\ Were those friends drunk? \\ 
        C be like, the proof is left as an exercise for the waiter)

        Let the first friend be A, the second friend be B, and the third friend be C. \\ 
        The waiter's question was that if everyone wanted cake. \\ 
        A replies ``I don't know''. Implying he doesn't know if everyone wanted cake. \\ 
        B replies ``I don't know''. Implying he doesn't know if everyone wanted cake. \\ 
        C replies ``No, not everyone wants cake'' implying that C did not want cake, as the question was if everyone wanted cake. \\ 
        Therefore, C does not get cake but A and B do get cake.
    \end{solution}
\end{questions}

\begin{problem}{6}
Sheikh Chilly, famous for his bizarre sense of humor and love of logic puzzles, left the following clues
regarding the location of the hidden treasure. The treasure can only be in one place.
\begin{itemize}
    \item If the house is next to a lake, then the treasure is in the kitchen.
    \item If the house is not next to a lake or the treasure is buried under the flagpole, then the tree in the front yard is an elm and the tree in the back yard is not an oak.
    \item If the treasure is in the garage, then the tree in the back yard is not an oak.
    \item If the treasure is not buried under the flagpole, then the tree in the front yard is not an elm.
    \item The treasure is not in the kitchen.
\end{itemize}
Using rules of inference, determine where the treasure is hidden. Clearly state what your propositions
represent.
\end{problem}

\begin{questions}
    \question
    \begin{solution}
        Our propositions are as follows: \\ 
        $p$: ``The house is next to a lake'' \\ 
        $q$: ``The tresure is in the kitchen'' \\ 
        $r$: ``The treasure is buried under the flagpole'' \\ 
        $s$: ``The tree in the front yard is an elm'' \\ 
        $t$: ``The tree in the back yard is an oak'' \\ 
        $u$: ``The treasure is in the garage'' 

        The above situation in propositional logic can be represented as: \\ 
        1: $ p \implies q $ \\ 
        2: $ (\neg p \lor r) \implies (s \land \neg t) $ \\ 
        3: $ u \implies \neg t $ \\ 
        4: $ \neg r \implies \neg s $ \\ 
        5: $ \neg q $
        
        \vspace{2mm}
        Now we know that $ \neg q $ is true. \\ 
        Then $ \neg q \implies \neg p $ (By Contrapositive) \\ 
        So $ \neg p $ is true by Modus Tollens.
        
        Then: $ (\neg p \lor r) $ is true. \\ 
        $ \therefore s \land \neg t $ is true. \\ 
        $ \therefore s $ is true by Simplification \\ 
        $ \therefore \neg t $ is true by Simplification 

        Then by taking the contrapositive of proposition 4 \\ 
        $ s \implies r $ [A conditional statement has the same truth value as its contrapositive] \\ 
        Since $s \implies r$, and we know $s$ is true, therefore $r$ is true by Modus Ponens. 

        Hence we can conclude that the treasure is buried under the flagpole.
    \end{solution}
\end{questions}

\begin{problem}{7}
Show that following are logically equivalent without using truth tables
\begin{itemize}
    \item[(a)] $ (p \implies r) \lor (q \implies r)  \equiv  (p \land q) \implies r $
    \item[(b)] $ p \land (q \lor r) \equiv (p \land q) \lor (p \land) r $
    \item[(c)] $ \neg [\neg[(p \lor q) \land r] \lor \neg q] \equiv q \land r $
        \item[(d)]$ (p \lor q \lor r) \land (p \lor t \lor \neg q) \land (p \lor \neg t \lor r) \equiv p \lor [r \land (t \lor \neg q)] $
\end{itemize}

\end{problem}

\begin{questions}
    \question
    \begin{solution}
        
        (a) $ (p \implies r) \lor (q \implies r)  \equiv  (p \land q) \implies r $ \\ 
        LHS: \\ 
        $ \neg((p \implies r) \lor \neg(q \implies r)) $ \hspace{10mm} {\color{red} [$ \neg (A \lor B) \equiv \neg A \land \neg B $]}\\ 
        $ (p \land \neg r) \land (q \land \neg r) $ \hspace{21.5mm} {\color{red} [$ \neg (p \implies q) \equiv p \land \neg q $]}\\ 
        $ (p \land q) \land \neg r $ \hspace{31mm} { \color{red} [Distributive Law]}\\ 
        $ (p \land q) \implies r $ \hspace{27.5mm} { \color{red} [$ \neg (p\implies q) \equiv p \land \neg q $]} \\ 
        RHS: \\ 
        $ \neg ((p \land q) \implies r) $ \\ 
        $ (p \land q) \land \neg r $ \\ 
        $ (p \land \neg r) \land (q \land \neg r) $ \\ 
        $ \neg ((p \land \neg r) \land (q \land \neg r)) $ \\ 
        $ \neg (p \land \neg r) \lor \neg(q \land \neg r) $ \\ 
        $ \neg \neg (p \implies r) \lor \neg \neg (q \implies r) $ \\ 
        $ (p \implies r) \lor (q \implies r) $ 

        \vspace{2mm}
        (b)
    \end{solution}
\end{questions}

\begin{problem}{8}
Use Truth tables to see if the following statements are logically equivalent.
\begin{itemize}
    \item [(a)] $ p \implies (q \lor r) \equiv (q \implies p) \land (p \implies r) $
    \item [(b)] $ (p \lor q) \implies r \equiv (p \implies r) \land (q \implies r) $
    \item [(c)] $ p \implies (q \lor r) \equiv \neg r \implies ( p \implies q) $
\end{itemize}

\end{problem}

\begin{questions}
    \question
    \begin{solution}
        
    \end{solution}

\end{questions}
\end{sloppypar}
\end{document}