\documentclass[addpoints]{exam}

\usepackage{amsmath,amssymb,amsthm}
\usepackage{tabularx}
\usepackage{tikz}
\usetikzlibrary{graphs,graphs.standard}
\usepackage{xcolor}

% \usepackage{draftwatermark}
% \SetWatermarkText{Sample Solution}
% \SetWatermarkScale{3}

\theoremstyle{definition}
\newtheorem{definition}{Definition}[section]

\theoremstyle{claim}
\newtheorem{claim}{Claim}

\title{Quiz 1B: Parity}
\author{CS/MATH 113 Discrete Mathematics L1}
\date{Habib University | Spring 2023}

\printanswers

\begin{document}
\maketitle
\thispagestyle{empty}
\noindent
\begin{tabularx}{\linewidth}{Xr}
  Total Marks: \numpoints & Date: \today\\
  Duration: 15 minutes & Time: 1715--1730h
\end{tabularx}
\hrule
\bigskip

\noindent \textbf{Student ID}: \hrulefill \\[5pt]
\noindent \textbf{Student Name}: \hrulefill \\[5pt]

\section{Problems}

\begin{questions}
\question[5] Given the following definitions, prove the claim below.

\begin{definition}[Even integer]
  An integer is \textit{even} if it can be written as $2k$ where $k$ is an integer.
\end{definition}

\begin{definition}[Odd integer]
  An integer is \textit{odd} if it can be written as $2k+1$ where $k$ is an integer.
\end{definition}

\begin{definition}[Parity]
  The \textit{parity} of an integer is its property of being even or odd.
\end{definition}

\begin{claim}
  Given integers, $m$ and $n$ of different parity, $m\cdot n$ is odd.
\end{claim}

  
  \begin{solution}
    There are 2 cases to consider: $m$ is even and $n$ is odd, and vice versa. The proof proceeds similarly in both cases, so only the case of even $m$ and odd $n$ is explored below.
    \begin{proof}
      Let $m=2p$ and $n=2q+1$ where $p$ and $q$ are integers.\\
      Then $mn=4pq+2p = 2 (2pq+p)$.\\
      $2pq+p$ is an integer.\\
      $\therefore mn$ is even.\\
      $\therefore$ the claim not true and cannot be proved.
    \end{proof}
  \end{solution}
\end{questions}
\end{document}